\section{Measurable Functions}

\begin{definition}
Let $(X,\A)$ and $(X',\A')$ be measurable spaces.
A function $f:X \to X'$ is said to be \textbf{measurable} if $f^{-1}(E) \in \A$ for every $E \in \A'$.
\end{definition}

\begin{remark}
Continuity and measurability are similar and related.
Given $(X,d)$ and $(X',d')$ metric spaces, let $\G$ be the collection of all open sets in $X$ and $\G'$ the collection of all open sets in $X'$.
Then $f:X \to X'$ is continuous if and only if $f^{-1}(E) \in \G$ for every $E \in \G'$.
\end{remark}

\begin{proposition}
Given $(X,\A)$, $(X',\A')$, and $(X'',\A'')$ measurable spaces, let $f:X \to X'$ and $g:X' \to X''$ be measurable.
Then $g \circ f:X \to X''$ is measurable.
\end{proposition}

\begin{theorem} [Characterization by Generating Sets]
Let $(X,\A)$ and $(X',\A')$ be measurable spaces.
Let $C' \subseteq \Pset(X')$ be such that $\sigma_0(C') = \A'$.
Then $f:X \to X'$ is measurable if and only if $f^{-1}(E) \in \A$ for every $E \in C'$.
\end{theorem}

\begin{remark}
The implication ($\Rightarrow$) is trivial since measurability is stronger than the characterizing condition, while ($\Leftarrow$) is useful.
Indeed, to check if a function is measurable we should consider all sets in $\A'$, but if there is a set $C'$ that generates the $\sigma$-algebra $\A'$, it is enough to consider all sets in $C'$, which are usually fewer than all sets of $\A'$.
Instead of considering the entire $\sigma$-algebra, we are just considering a small collection of sets.
\end{remark}

\begin{definition}
Consider $(X,\LL)$ a measurable space.
Consider $(X',d')$ a metric space, on which we consider $(X',\B')$ measurable space, where $\B'$ is the Borel $\sigma$-algebra.
Then $f:X \to X'$ is \textbf{Lebesgue measurable} if $f^{-1}(E) \in \LL$ for every $E \in \B'$.
\end{definition}

\begin{definition}
Let $(X,d)$ and $(X,\B)$, $(X',d')$ and $(X',\B')$ be metric spaces with their Borel $\sigma$-algebras.
Then $f:X \to X'$ is \textbf{Borel measurable} if $f^{-1}(E) \in \B$ for every $E \in \B'$.
\end{definition}

\begin{corollary}
Consider $(X,\LL)$ a measurable space and $(X',d')$ a metric space with $(X',\B')$ measurable space.
Then $f:X \to X'$ is Lebesgue measurable if and only if $f^{-1}(E) \in \LL$ for every open set $E \subseteq X'$.
\end{corollary}

\begin{proof}
Let $C' = \{\text{open sets of } X'\}$, then $\sigma_0(C') = \B'$.
By the preceding theorem, $f:X \to X'$ is $\LL$-measurable if and only if $f^{-1}(E) \in \LL$ for every $E \in \B'$.
\end{proof}

\begin{corollary}
Given $X$, $X'$ metric spaces, consider $f:X \to X'$ continuous.
Then $f$ is Borel measurable.
\end{corollary}

\begin{proof}
$f$ is continuous if and only if $f^{-1}(E) \in \G$ for every $E \in \G'$.
By the preceding theorem, instead of taking $\G'$ we take the $\sigma$-algebra:
$f^{-1}(E) \in \G$ for every $E \in \sigma_0(\G') = \B'$.
But $\G \subset \B$.
\end{proof}

\clearpage
\begin{corollary}
Given $X$, $X'$ metric spaces, consider $f:X \to X'$ Borel measurable.
Then $f$ is Lebesgue measurable.
\end{corollary}

\begin{proof}
We know $f^{-1}(E) \in \B$ for every $E \in \sigma_0(\G') = \B'$.
But $\B \subseteq \LL$.
\end{proof}

\begin{corollary}
Given $X$, $X'$ metric spaces, if $f:X \to X'$ is continuous, then $f$ is Lebesgue measurable.
\end{corollary}

\begin{theorem}
Let $f:X \subseteq \R \to \R$ be Lebesgue measurable and $g:\R \to \R$ be continuous.
Then $g \circ f$ is Lebesgue measurable.
\end{theorem}

\begin{proof}
We have $(X \subseteq \R, \LL(\R))$ and $(\R, \B(\R))$.
We know:
\begin{itemize}
    \item $f$ is $\LL$-measurable if and only if $f^{-1}(E') \in \LL(\R)$ for every $E' \subseteq \R$ (Borel set).
    \item For every $E \subseteq \R$, $g^{-1}(E)$ is open if and only if $g$ is continuous.
\end{itemize}
Then for every $E \subseteq \R$:
\[
(g \circ f)^{-1}(E) = f^{-1}[g^{-1}(E)].
\]
Since $g$ is continuous, $g^{-1}(E) = E'$ is open.
Thus, $(g \circ f)^{-1}(E) = f^{-1}(E') \in \LL(\R)$.
\end{proof}

\begin{remark}[1]
If $g:\R \to \R$ is only Lebesgue measurable, then we arrive at $f^{-1}[g^{-1}(E)]$.
But now $g^{-1}(E) \in \LL(\R) \supsetneq \B(\R)$, and we cannot conclude $f^{-1}[g^{-1}(E)] \in \LL(\R)$.
So $f$, $g$ Lebesgue measurable does not imply $g \circ f$ is Lebesgue measurable.
\end{remark}

\begin{remark}[2]
Instead of continuity, we can also assume Borel measurability.
If $g:\R \to \R$ is Borel measurable, then we have $(\R, \B(\R))$ and $(\R, \B(\R))$.
As before, we get $f^{-1}[g^{-1}(E)]$, and now $g^{-1}(E) \in \B(\R)$, so we can conclude $f^{-1}[g^{-1}(E)] \in \LL(\R)$.
Thus, $g \circ f$ is Lebesgue measurable.
\end{remark}

\begin{theorem}[Lusin]
Let $f:E \subseteq \R \to \R$ be Lebesgue measurable.
Then for every $\varepsilon > 0$ there exist a continuous function $g:\R \to \R$ and a closed set $F \subseteq E$ such that:
\begin{itemize}
    \item $f = g$ in $F$,
    \item $\lambda(E \setminus F) < \varepsilon$.
\end{itemize}
\end{theorem}

\begin{theorem}
Let $(X,\A,\mu)$ be a complete measure space. Let $f,g:X \to \R$.
If $f$ is Lebesgue measurable and $f = g$ a.e., then $g$ is Lebesgue measurable.
\end{theorem}

\clearpage
\subsection{Real Valued Functions}

Define:
\[
\M(X,\A) \coloneqq \{ f:X \to \Rext \text{ measurable} \}.
\]
Notation: $(X,\A) = (X,\A)$, $(\Rext, \B(\Rext)) = (X',\A')$.

Also define:
\[
\M_+(X,\A) \coloneqq \{ f:X \to \Rext \text{ measurable}, f \geq 0 \}.
\]

For every $\alpha \in \R$:
\begin{align*}
\{f > \alpha\} &\coloneqq \{ x \in X : f(x) > \alpha \} = f^{-1}((\alpha, +\infty]), \\
\{f \geq \alpha\} &\coloneqq \{ x \in X : f(x) \geq \alpha \} = f^{-1}([\alpha, +\infty]), \\
\{f < \alpha\} &\coloneqq f^{-1}([-\infty, \alpha)), \\
\{f \leq \alpha\} &\coloneqq f^{-1}([-\infty, \alpha]).
\end{align*}

\begin{theorem}
Let $(X,\A)$ be a measurable space, $f:X \to \Rext$.
The following statements are equivalent:
\begin{enumerate}
    \item $f$ is measurable.
    \item $\{f > \alpha\} \in \A$ for every $\alpha \in \R$.
    \item $\{f \geq \alpha\} \in \A$ for every $\alpha \in \R$.
    \item $\{f < \alpha\} \in \A$ for every $\alpha \in \R$.
    \item $\{f \leq \alpha\} \in \A$ for every $\alpha \in \R$.
\end{enumerate}
\end{theorem}

Let $f,g:X \to \Rext$. Define:
\begin{align*}
\{f < g\} &\coloneqq \{ x \in X : f(x) < g(x) \}, \\
\{f \leq g\} &\coloneqq \{ x \in X : f(x) \leq g(x) \}, \\
\{f = g\} &\coloneqq \{ x \in X : f(x) = g(x) \}.
\end{align*}

\begin{theorem}
Let $f,g \in \M(X,\A)$. Then:
\begin{enumerate}
    \item $\{f < g\} \in \A$,
    \item $\{f \leq g\} \in \A$,
    \item $\{f = g\} \in \A$.
\end{enumerate}
\end{theorem}

\begin{theorem}
Let $\{f_n\} \subseteq \M(X,\A)$. Then:
\[
\sup_{n \in \N} f_n, \quad \inf_{n \in \N} f_n \in \M(X,\A).
\]
\end{theorem}

\begin{proof}
For every $\alpha \in \R$:
\[
\left\{ \sup_{n \in \N} f_n > \alpha \right\} = \bigcup_{n=1}^{\infty} \{ f_n > \alpha \} \in \A.
\]
For the infimum, note that:
\[
\inf_{n \in \N} f_n = -\sup_{n \in \N} (-f_n).
\]
\end{proof}

\begin{corollary}
If $f,g \in \M(X,\A)$, then:
\[
\max\{f,g\}, \quad \min\{f,g\}, \quad f_{\pm} \in \M(X,\A).
\]
\end{corollary}

\begin{theorem}
Let $\{f_n\} \subseteq \M(X,\A)$. Then:
\[
\liminf_{n \to \infty} f_n, \quad \limsup_{n \to \infty} f_n \in \M(X,\A).
\]
\end{theorem}

\begin{proof}
Note that:
\[
\limsup_{n \to \infty} f_n = \inf_{k \geq 1} \left( \sup_{n \geq k} f_n \right) \in \M(X,\A),
\]
and
\[
\liminf_{n \to \infty} f_n = -\limsup_{n \to \infty} (-f_n) \in \M(X,\A).
\]
\end{proof}

\begin{theorem}
Let $f,g:X \to \R$, with $f,g \in \M(X,\A)$. Then:
\[
f + g \in \M(X,\A), \quad fg \in \M(X,\A).
\]
\end{theorem}

\subsection{The Cantor Set}

\subsubsection{Construction}

The Cantor set is constructed through an iterative process:

\begin{itemize}
    \item \textbf{Step 0:} Start with $K_0 \coloneqq [0,1]$
    \item \textbf{Step 1:} Remove the open middle third $I_{0,1} \coloneqq (1/3, 2/3)$, leaving:
    \[
    K_1 = [0,1/3] \cup [2/3,1]
    \]
    \item \textbf{Step n:} After $n$ steps, we have $2^n$ closed intervals $J_{n,k}$ of length $1/3^n$
    \item \textbf{Step n+1:} From each $J_{n,k}$, remove the open middle third $I_{n,k}$ of length $1/3^{n+1}$
\end{itemize}

Define the sets at each stage:
\[
K_n \coloneqq \bigcup_{k=1}^{2^n} J_{n,k}, \quad K \coloneqq \bigcap_{n=0}^{\infty} K_n
\]
The set $K$ is the \textbf{Cantor set}.

\subsubsection{Basic Properties}

\begin{itemize}
    \item \textbf{Closedness:} Each $K_n$ is a finite union of closed intervals, hence closed. Since $K$ is an intersection of closed sets, $K$ is closed.
    \item \textbf{Measurability:} $K$ is closed $\Rightarrow K \in \B(\R) \subseteq \LL(\R)$
    \item \textbf{Total length removed:} Define the removed set:
    \[
    \Omega \coloneqq \bigcup_{n=0}^{\infty} \bigcup_{k=1}^{2^n} I_{n,k}
    \]
    Since $I_{n,k}$ are open, $\Omega$ is open and measurable. The total length removed is:
    \[
    \lambda(\Omega) = \sum_{n=0}^{\infty} \frac{2^n}{3^{n+1}} = \frac{1}{3} \sum_{n=0}^{\infty} \left( \frac{2}{3} \right)^n = \frac{1}{3} \cdot \frac{1}{1 - 2/3} = 1
    \]
    Hence, $\lambda(K) = \lambda([0,1]) - \lambda(\Omega) = 0$
\end{itemize}

\subsubsection{Topological Properties}

\begin{itemize}
    \item \textbf{Uncountability:} $K$ is uncountable (can be shown via ternary expansion)
    \item \textbf{Empty interior:} $\operatorname{int}(K) = \emptyset$. Any open interval contained in $K$ would have positive measure, but $\lambda(K) = 0$
    \item \textbf{Density of complement:} $\overline{[0,1] \setminus K} = [0,1]$. For any $x_0 \in [0,1]$ and $r > 0$, the interval $(x_0 - r, x_0 + r)$ cannot be contained in $K$ (since $\operatorname{int}(K) = \emptyset$), so it must intersect $[0,1] \setminus K$
    \item \textbf{Perfect set:} $K$ equals its set of accumulation points. For any $\bar{x} \in K$ and $\varepsilon > 0$, choose $n$ such that $1/3^n < \varepsilon$. Since $\bar{x} \in J_{n,k}$ for some $k$, the removed interval $I_{n,k} \subseteq (\bar{x} - \varepsilon, \bar{x} + \varepsilon)$ has endpoints in $K$ different from $\bar{x}$
\end{itemize}

\subsection{The Vitali-Lebesgue Function}

\subsubsection{Construction}

Define a sequence of functions $L_n:[0,1] \to [0,1]$ recursively:

\begin{itemize}
    \item \textbf{Base:} $L_0(x) = x$ (the identity function)
    \item \textbf{Inductive step:} On each interval $J_{n,k}$ of $K_n$, define $L_{n+1}$ to be linear with slope $(3/2)^{n+1}$, preserving continuity
    \item On the removed intervals $[0,1] \setminus K_n$, keep $L_{n+1}$ constant (equal to its values at the endpoints)
\end{itemize}

Explicitly:
\[
L_n(x) \coloneqq 
\begin{cases}
\text{linear with slope } (3/2)^n & \text{on each } J_{n,k} \subset K_n \\
\text{constant} & \text{on } [0,1] \setminus K_n
\end{cases}
\]

\subsubsection{Properties of the Sequence}

\begin{itemize}
    \item \textbf{Symmetry:} $L_n(x) + L_n(1-x) = 1$ for all $n \in \N$, $x \in [0,1]$
    \item \textbf{Monotonicity:} $L_{n+1}(x) \leq L_n(x)$ for all $n \in \N$, $x \in [0,1]$
    \item \textbf{Uniform convergence:} The maximum difference between successive functions is:
    \[
    \sup_{x \in [0,1]} |L_{n+1}(x) - L_n(x)| = \frac{1}{2^{n+1}}
    \]
    This follows from calculating the maximum difference on the first interval $J_{n,1} = [0, 1/3^n]$:
    \[
    |L_{n+1}(x) - L_n(x)| = \left( \frac{3}{2} \right)^n \left( \frac{3}{2} - 1 \right) x = \frac{1}{2^{n+1}} x
    \]
    Hence, $\{L_n\}$ is uniformly Cauchy and converges uniformly to a function $L$
\end{itemize}

\subsubsection{Properties of the Limit Function}

The \textbf{Vitali-Lebesgue function} is defined as:
\[
L(x) \coloneqq \lim_{n \to \infty} L_n(x)
\]

\begin{itemize}
    \item \textbf{Continuity:} $L \in C([0,1])$ (uniform limit of continuous functions)
    \item \textbf{Monotonicity:} $L$ is nondecreasing
    \item \textbf{Symmetry:} $L(x) + L(1-x) = 1$
    \item \textbf{Singular derivative:} $L'(x) = 0$ almost everywhere. On each $[0,1] \setminus K_n$, $L$ is constant, so $L' = 0$. Since $\lambda(K) = 0$, $L' = 0$ a.e.
    \item \textbf{Non-constant:} Despite zero derivative a.e., $L(1) - L(0) = 1 - 0 = 1$
\end{itemize}

\begin{remark}
The Vitali-Lebesgue function is a classic example of a \textbf{singular function}: continuous, nondecreasing, with derivative zero almost everywhere, yet not constant. This demonstrates that the Fundamental Theorem of Calculus requires absolute continuity, not just continuity.
\end{remark}

\subsection{Applications}

\subsubsection{Application One}
Consider a continuous function $f:D \subseteq \R \to \R$.
In general, given $E \subseteq \R$ with $E \in \LL(\R)$, can we say $f^{-1}(E) \in \LL(\R)$? The answer is no.

\begin{proof}
Define:
\[
h(x) \coloneqq \frac{x + L(x)}{2}, \quad x \in [0,1].
\]
Then $h$ is continuous, strictly increasing, and surjective.
Moreover, $h([0,1] \setminus K)$ is a union of open intervals with total length:
\[
\lambda(h([0,1] \setminus K)) = \frac{1}{6} \sum_{k=0}^{\infty} \left( \frac{2}{3} \right)^k = \frac{1}{2}.
\]
Since $[0,1] = h([0,1]) = h(K) \cup h([0,1] \setminus K)$, we have $\lambda(h(K)) = \frac{1}{2}$.

There exists $\tilde{E} \subseteq h(K)$ such that $\tilde{E} \notin \LL(\R)$ (e.g., a Vitali set).
On the other hand, $h^{-1}(\tilde{E}) \subseteq K$, so $h^{-1}(\tilde{E}) \in \LL(\R)$ since $K$ has measure zero and the Lebesgue $\sigma$-algebra is complete.

Now consider $f \coloneqq h^{-1}$ and $\tilde{F} \coloneqq f(\tilde{E}) = h^{-1}(\tilde{E}) \in \LL(\R)$.
Since $f$ is continuous, but $\tilde{E} = f^{-1}(\tilde{F}) \notin \LL(\R)$, we obtain the counterexample.
\end{proof}

\clearpage
\subsubsection{Application Two}
$(\R, \B(\R), \lambda) \subsetneq (\R, \LL(\R), \lambda)$.

\begin{proof}
From above, $\tilde{F} \in \LL(\R)$ and $f$ is Lebesgue measurable (since continuous).
If $\tilde{F} \in \B(\R)$, then $f^{-1}(\tilde{F}) = \tilde{E} \in \LL(\R)$, which is false.
Hence, $\tilde{F} \notin \B(\R)$.
\end{proof}

\subsubsection{Application Three}
If $f$ is Borel measurable and $g$ is Lebesgue measurable, then we can't say $g \circ f$ is Lebesgue measurable for sure.

\begin{proof}
Let $f \coloneqq h^{-1}$ as above.
Take $\tilde{F} \in \LL(\R)$ and define $g \coloneqq \chi_{\tilde{F}}$, the indicator function of $\tilde{F}$.
Then $g$ is Lebesgue measurable.
Now:
\[
(g \circ f)^{-1}((1/2, 3/2)) = f^{-1}(g^{-1}((1/2, 3/2))) = f^{-1}(\tilde{F}) = \tilde{E} \notin \LL(\R).
\]
Hence, $g \circ f$ is not Lebesgue measurable.
\end{proof}

\subsubsection{Application Four}
$(\R, \B(\R), \lambda)$ is not complete.

\begin{proof}
The Cantor set $K$ is closed, hence Borel, and $\lambda(K) = 0$.
There exists $\tilde{F} \subseteq K$ such that $\tilde{F} \notin \B(\R)$ (as above).
Hence, the Borel measure space is not complete.
\end{proof}

\subsubsection{Application Five}
Define $f:\R \to \R$ by:
\[
f(x) \coloneqq 
\begin{cases}
2 & x \in \tilde{F} \subset K, \\
1 & x \in K \setminus \tilde{F}, \\
0 & x \in \R \setminus K.
\end{cases}
\]
Then $f = g$ a.e. where $g \equiv 0$. $g$ is Borel measurable (and Lebesgue measurable).
But $\{f \geq 2\} = \tilde{F} \notin \B(\R)$, so $f$ is not Borel measurable.
This shows the need for completeness to pass measurability a.e.

\subsection{Measurable Functions: Further Properties}

\begin{corollary}
Let $f:X \to \R$.
\begin{enumerate}
    \item $f \in \M(X,\A)$ if and only if $f_{\pm} \in \M(X,\A)$.
    \item $f \in \M(X,\A)$ implies $|f| \in \M(X,\A)$.
\end{enumerate}
\end{corollary}

\begin{proof}
(1) The forward implication is known. For the converse, note $f = f_+ - f_- \in \M(X,\A)$.

(2) If $f \in \M(X,\A)$, then $f_{\pm} \in \M(X,\A)$ by (1), so $|f| = f_+ + f_- \in \M(X,\A)$.
\end{proof}

\clearpage
\begin{lemma}
Given $C \subseteq X$, the indicator function $\chi_C \in \M(X,\A)$ if and only if $C \in \A$.
\end{lemma}

\begin{proof}
We have:
\[
\{\chi_C > \alpha\} = 
\begin{cases}
X \in \A & \alpha < 0, \\
C \in \A & \alpha \in [0,1), \\
\emptyset \in \A & \alpha \geq 1.
\end{cases}
\]
Hence, $\chi_C \in \M(X,\A) \Leftrightarrow C \in \A$.
\end{proof}

\begin{remark}
In general, $|f| \in \M(X,\A)$ does not imply $f \in \M(X,\A)$.
For example, let $E \subseteq X$ with $E \notin \A$.
Define:
\[
f(x) \coloneqq \chi_E(x) - \chi_{E^c}(x) = 
\begin{cases}
1 & x \in E, \\
-1 & x \in E^c.
\end{cases}
\]
Then $\{f > 1/2\} = E \notin \A$, so $f \notin \M(X,\A)$.
But $|f| \equiv 1 \in \M(X,\A)$.
\end{remark}

\subsection{Simple Functions}

\begin{definition}
Let $X$ be a set. A function $s:X \to \R$ is called a \textbf{simple function} if $s(X)$ is finite.
\end{definition}

If $s(X) = \{c_1, c_2, \dots, c_n\}$ with $c_k$ distinct, define $E_k \coloneqq \{ x \in X : s(x) = c_k \}$.
Then the \textbf{canonical form} of $s$ is:
\[
s = \sum_{k=1}^{n} c_k \chi_{E_k}.
\]
The sets $\{E_k\}_{k=1}^n$ form a partition of $X$.

\begin{remark}
$s \in \M(X,\A)$ if and only if $E_k \in \A$ for $k = 1,\dots,n$.
\end{remark}

Define:
\begin{align*}
\Scal(X,\A) &\coloneqq \{ \text{measurable simple functions } f:X \to \R \}, \\
\Scal_+(X,\A) &\coloneqq \{ \text{measurable simple functions } f:X \to \R, f \geq 0 \}.
\end{align*}

\begin{remark}
If $(X,\A) = (\R, \LL(\R))$ and each $E_k$ is an interval, then $f$ is a step function.
\end{remark}

\clearpage
\begin{theorem}[Simple Approximation Theorem]
Let $(X,\A)$ be a measurable space and $f:X \to \Rext$.
Then there exists a sequence $\{s_n\}$ of simple functions such that $s_n \to f$ pointwise on $X$ as $n \to \infty$.
Furthermore:
\begin{enumerate}
    \item If $f \in \M(X,\A)$, then $\{s_n\} \subseteq \Scal(X,\A)$.
    \item If $f \geq 0$, then $\{s_n\}$ is increasing and $0 \leq s_n \leq f$.
    \item If $f$ is bounded, then $s_n \to f$ uniformly on $X$.
\end{enumerate}
\end{theorem}

\begin{proof} [Sketch]
For $f \geq 0$ bounded, say $0 \leq f \leq 1$, divide $[0,1]$ into $2^n$ intervals of length $2^{-n}$.
Define:
\[
E_k^{(n)} \coloneqq \left\{ x \in X : \frac{k}{2^n} \leq f(x) < \frac{k+1}{2^n} \right\}, \quad k = 0,1,\dots,2^n-1.
\]
Then set:
\[
s_n \coloneqq \sum_{k=0}^{2^n-1} \frac{k}{2^n} \chi_{E_k^{(n)}}.
\]
The sequence $\{s_n\}$ has the desired properties.
\end{proof}

\subsection{Essentially Bounded Functions}

Let $(X,\A,\mu)$ be a measure space. For any $N \in \NN_\mu$ (null sets), define:
\[
\alpha_N \coloneqq \sup_{x \in N^c} f(x).
\]
Note: if $N_2 \subseteq N_1$, then $\alpha_{N_2} \geq \alpha_{N_1}$.

\begin{definition}
The \textbf{essential supremum} of $f$ is:
\[
\esssup_X f \coloneqq \inf \left\{ \sup_{x \in N^c} f(x) : N \in \NN_\mu \right\}.
\]
The \textbf{essential infimum} is:
\[
\essinf_X f \coloneqq \sup \left\{ \inf_{x \in N^c} f(x) : N \in \NN_\mu \right\}.
\]
\end{definition}

\begin{proposition}
Let $f$ be measurable. Then there exists $N \in \NN_\mu$ such that:
\[
\esssup_X f = \sup_{x \in N^c} f(x).
\]
Moreover, $f(x) \leq \esssup_X f$ almost everywhere.
\end{proposition}

\clearpage
\subsubsection{Properties}
\begin{enumerate}
    \item If $f \in \M(X,\A)$, then:
    \[
    \esssup_X f = - \essinf_X (-f), \quad \esssup_X (kf) = k \esssup_X f \text{ for } k \geq 0.
    \]
    \item If $f,g \in \M(X,\A)$, then:
    \begin{enumerate}
        \item $f \leq g$ a.e. $\Rightarrow$ $\esssup_X f \leq \esssup_X g$.
        \item $\esssup_X (f+g) \leq \esssup_X f + \esssup_X g$.
        \item $f = g$ a.e. $\Rightarrow$ $\esssup_X f = \esssup_X g$.
        \item If $g \geq 0$ a.e., then $fg \leq (\esssup_X f) g$ a.e.
    \end{enumerate}
\end{enumerate}

\begin{definition}
A function $f \in \M(X,\A)$ is \textbf{essentially bounded} if:
\[
\esssup_X |f| < \infty.
\]
Define:
\[
\LL^\infty(X,\A,\mu) \coloneqq \{ f:X \to \Rext \text{ such that } f \text{ is essentially bounded} \}.
\]
\end{definition}

\begin{remark}
Note: $L^\infty(X,\A,\mu) \neq \LL^\infty(X,\A,\mu)$ (the latter is usually the space of equivalence classes).
\end{remark}

\begin{remark}
\begin{enumerate}
    \item If $f \in \LL^\infty$, then $f$ is finite a.e., since $|f| \leq \esssup_X f < \infty$ a.e.
    \item $f$ finite a.e. does not imply $f \in \LL^\infty$.
\end{enumerate}
\end{remark}

\begin{example}
Let $f:\R \to \Rext$ be defined by:
\[
f(x) \coloneqq 
\begin{cases}
1/|x| & x \neq 0, \\
+\infty & x = 0.
\end{cases}
\]
Then $f$ is finite on $\R \setminus \{0\}$, so finite a.e., but $\esssup_X f = +\infty$.
\end{example}

\begin{example}
Let:
\[
f(x) \coloneqq 
\begin{cases}
\arctan(x) & x \in \R \setminus A, \\
+\infty & x \in A,
\end{cases}
\]
where $A = \{1/n : n \in \N\}$.
Then $\sup_X f = +\infty$, but $\esssup_X f = \pi/2$ since $A$ is countable (measure zero).
So $f \in \LL^\infty$, but $f$ is not bounded in the usual sense.
\end{example}