\section{Measure Theory}

\subsection{$\sigma$-Algebra and Measurable Space}
Let $X$ be a set.

\begin{definition}
A family $\A \subseteq \Pset(X)$ is said to be a \textbf{$\sigma$-algebra} if
\begin{enumerate}
    \item $\emptyset \in \A$,
    \item $E \in \A \Rightarrow E^C \in \A$,
    \item $\{E_k\}_{k \in \N} \subseteq \A \Rightarrow \bigcup_{k=1}^{\infty} E_k \in \A$.
\end{enumerate}
The pair $(X, \A)$ is called a \textbf{measurable space} when $\A$ is a $\sigma$-algebra.
The elements of $\A$ are called \textbf{measurable sets}.
\end{definition}

\begin{theorem}
Let $S \subseteq \Pset(X)$; then there exists a $\sigma$-algebra $\sigma_0(S)$ such that
\begin{enumerate}
    \item $S \subseteq \sigma_0(S)$,
    \item For every $\sigma$-algebra $\A \subseteq \Pset(X)$ with $\A \supseteq S$, we have $\A \supseteq \sigma_0(S)$.
\end{enumerate}
This $\sigma_0(S)$ is called the \textbf{$\sigma$-algebra generated by $S$}, and it is the minimal such $\sigma$-algebra.
\end{theorem}

\begin{remark}
A $\sigma$-algebra is a collection of subsets closed under complementation and countable unions. The generated $\sigma$-algebra $\sigma_0(S)$ is the smallest $\sigma$-algebra containing $S$. This is a fundamental concept because it allows us to define the smallest family of sets where a measure can be defined.
\end{remark}

\subsection{Borel Sets}

\begin{definition}
Let $(X,d)$ be a metric space. Let $\G$ (called \textbf{topology}) be a family of open sets in $X$.
The $\sigma$-algebra $\sigma_0(\G)$ is the \textbf{Borel $\sigma$-algebra}, denoted by $\B(X)$.
The elements of $\sigma_0(\G)$ are called \textbf{Borel sets}.
\end{definition}

\begin{remark}
The following sets are Borel: open sets, closed sets, countable intersections of open sets, countable unions of closed sets.
\end{remark}

\begin{example}
We know that $\B(\R) = \sigma_0(\G)$, but we can use other families instead of open sets:
$\B(\R) = \sigma_0(\G) = \sigma_0(I)$, where
$I = \{(a,b) : a,b \in \R, a \leq b\}$ or
$I = \{[a,b] : a,b \in \R, a \leq b\}$ or
$I = \{(a,b] : a,b \in \R, a \leq b\}$.

Also, $\B(\Rn) = \sigma_0(K)$, where $K$ is the family of $N$-dimensional closed or open rectangles.

Defining $\Rext = \R \cup \{-\infty, +\infty\}$, we have $\B(\Rext) = \sigma_0(I)$, where $I = \{(a, +\infty] : a \in \R\}$.
\end{example}

\begin{remark}
In metric spaces, the Borel $\sigma$-algebra is generated by open sets. It includes many familiar sets like open, closed, and sets constructed from them via countable operations. This is the natural domain for measures in topological spaces.
\end{remark}

\subsection{Measure}
Let $X$ be a set, and let $\A$ be a $\sigma$-algebra.

\begin{definition}
A function $\mu: \A \to \Rpext$ is a \textbf{measure} if
\begin{enumerate}
    \item $\mu(\emptyset) = 0$,
    \item For every disjoint family $\{E_k\}_{k \in \N} \subseteq \A$, we have $\mu\left(\bigcup_{k=1}^{\infty} E_k\right) = \sum_{k=1}^{\infty} \mu(E_k)$.
    This is called \textbf{$\sigma$-additivity}.
\end{enumerate}
\end{definition}

\begin{definition}
A measure $\mu$ is \textbf{finite} if $\mu(X) < \infty$.
A measure $\mu$ is \textbf{$\sigma$-finite} if there exists $\{E_k\}_{k \in \N} \subseteq \A$ such that $X = \bigcup_{k=1}^{\infty} E_k$ and $\mu(E_k) < \infty$ for every $k \in \N$.
\end{definition}

\begin{definition}
Let $\A \subseteq \Pset(X)$ be a $\sigma$-algebra and $\mu: \A \to \Rpext$ be a measure.
$(X, \A, \mu)$ is called a \textbf{measure space}.
If $\mu(X) = 1$, then $(X, \A, \mu)$ is a \textbf{probability space} and $\mu$ is a \textbf{probability measure}.
\end{definition}

\begin{example}
The \textbf{counting measure}:
$\mu: \Pset(X) \to \Rpext$ defined by
\[
\mu(E) \coloneqq
\begin{cases}
|E| & \text{if } E \subseteq X \text{ is finite} \\
\infty & \text{otherwise}
\end{cases}
\]
where $|E|$ is the cardinality.
It is finite if and only if $X$ is finite, and $\sigma$-finite if and only if $X$ is countable.

The \textbf{Dirac measure} $\delta_{x_0}$, concentrated at $x_0$:
Take $X \neq \emptyset$, $x_0 \in X$, define $\delta_{x_0}: \Pset(X) \to \Rpext$ by
\[
\delta_{x_0}(E) \coloneqq
\begin{cases}
1 & \text{if } x_0 \in E \\
0 & \text{otherwise}
\end{cases}
\]
\end{example}

\begin{remark}
Measures assign a "size" to sets in a consistent way. Key examples include the counting measure (number of elements) and the Dirac measure (concentrated at a point). Finite and $\sigma$-finite measures are important for avoiding pathological behavior.
\end{remark}

\clearpage
\subsection{Properties of Measures}

\begin{theorem}
Let $(X, \A, \mu)$ be a measure space. Then
\begin{enumerate}
    \item \textbf{Finite additivity}: For every finite disjoint family $\{E_1, \dots, E_n\} \subseteq \A$,
    $\mu\left(\bigcup_{k=1}^{n} E_k\right) = \sum_{k=1}^{n} \mu(E_k)$.

    \item \textbf{Monotonicity}: For every $E, F \in \A$ with $E \subseteq F$, we have $\mu(E) \leq \mu(F)$.

    \item \textbf{$\sigma$-subadditivity}: For every countable family $\{E_k\} \subset \A$ (not necessarily disjoint),
    $\mu\left(\bigcup_{k=1}^{\infty} E_k\right) \leq \sum_{k=1}^{\infty} \mu(E_k)$.

    \item \textbf{Continuity from below}: For every increasing sequence $\{E_k\} \subseteq \A$ (i.e., $E_k \nearrow$),
    
     $\mu\left(\bigcup_{k=1}^{\infty} E_k\right) = \mu\left(\lim_{k \to \infty} E_k\right) = \lim_{k \to \infty} \mu(E_k)$.

    \item \textbf{Continuity from above}: For every decreasing sequence $\{E_k\} \subseteq \A$ (i.e., $E_k \searrow$) with $\mu(E_1) < \infty$,
    
     $\mu\left(\bigcap_{k=1}^{\infty} E_k\right) = \mu\left(\lim_{k \to \infty} E_k\right) = \lim_{k \to \infty} \mu(E_k)$.
\end{enumerate}
\end{theorem}

\begin{remark}
Measures are well-behaved: they are additive, monotone, and continuous with respect to increasing/decreasing sequences. These properties make them suitable for integration and limit processes.
\end{remark}

\subsection{Sets of Zero Measure}
Let $(X, \A, \mu)$ be a measure space.

\begin{definition}
A set $N \subseteq X$ is said to be a \textbf{set of zero measure} if $N \in \A$ and $\mu(N) = 0$.
A set $E \subseteq X$ is \textbf{negligible} if there exists $N \in \A$ such that $E \subseteq N$ and $\mu(N) = 0$.

Let $\NN_\mu$ be the collection of sets of zero measure and $\T_\mu$ be the collection of negligible sets.
\end{definition}

\begin{example}
$X = \{a,b,c\}$, $\A = \{\emptyset, \{a\}, \{b,c\}, X\}$ is a $\sigma$-algebra.
Define $\mu(X) = \mu(\{a\}) \coloneqq 1$, $\mu(\emptyset) = \mu(\{b,c\}) \coloneqq 0$. This is a measure.

Then $N = \{b,c\} \in \NN_\mu$, and $\{b\}, \{c\} \subseteq N$ so $\{b\}, \{c\} \in \T_\mu \setminus \NN_\mu$.
\end{example}

\begin{definition}
A measure space $(X, \A, \mu)$ is said to be \textbf{complete} if $\T_\mu \subseteq \A$.
In such a case, $\mu$ is a \textbf{complete measure} and $\A$ is a \textbf{complete $\sigma$-algebra}.
\end{definition}

\begin{remark}
$\T_\mu = \NN_\mu$ if and only if $(X, \A, \mu)$ is complete.
\end{remark}

\begin{remark}
A complete measure space is one where all subsets of zero-measure sets are measurable. Completeness is desirable because it ensures that sets that are "tiny" are also measurable.
\end{remark}

\clearpage
\subsection{Almost Everywhere}
Consider a measure space $(X, \A, \mu)$.

\begin{definition}
A property $P$ on $X$ is said to be true \textbf{almost everywhere (a.e.)} if the set $\{x \in X : P(x) \text{ is false}\} \in \NN_\mu$.
\end{definition}

\begin{example}
\begin{itemize}
    \item $f, g: X \to \Rext$ are equal a.e. if $\{x \in X : f(x) \neq g(x)\} \in \NN_\mu$.
    \item $f: X \to \Rext$ is finite a.e. if $\{x \in X : f(x) = \pm \infty\} \in \NN_\mu$.
    \item $f: D \to \R$, with $D \in \A$, is said to be defined a.e. if $D^C \in \NN_\mu$.
\end{itemize}
\end{example}

\begin{remark}
Equal a.e. is an equivalence relation in the set of functions $f: X \to \Rext$.
\end{remark}

\subsection{Completion of a Measure Space}
Let $(X, \A, \mu)$ be a measure space.

Define
\[
\overline{\A} \coloneqq \{E \subseteq X : \exists F, G \in \A \text{ such that } F \subseteq E \subseteq G \text{ and } \mu(G \setminus F) = 0\}
\]
and $\overline{\mu}: \overline{\A} \to \Rpext$ by $\overline{\mu}(E) \coloneqq \mu(F)$.

\begin{theorem}
Let $(X, \A, \mu)$ be a measure space. Then
\begin{enumerate}
    \item $\overline{\A}$ is a $\sigma$-algebra with $\overline{\A} \supseteq \A$,
    \item $\overline{\mu}$ is a complete measure extending $\mu$, i.e., $\overline{\mu}|_{\A} = \mu$.
\end{enumerate}
The space $(X, \overline{\A}, \overline{\mu})$ is the \textbf{completion} of $(X, \A, \mu)$, and it is the smallest complete measure space containing $(X, \A, \mu)$.
\end{theorem}

\begin{remark}
The completion of a measure space is the smallest complete extension. It ensures that all subsets of zero-measure sets are included in the $\sigma$-algebra.
\end{remark}

\subsection{Outer Measure}
Let $X$ be a set.

\begin{definition}
A function $\mu^*: \Pset(X) \to \Rpext$ is said to be an \textbf{outer measure} on $X$ if
\begin{enumerate}
    \item $\mu^*(\emptyset) = 0$,
    \item $E_1 \subseteq E_2 \Rightarrow \mu^*(E_1) \leq \mu^*(E_2)$,
    \item \textbf{$\sigma$-subadditivity}: $\mu^*\left(\bigcup_{n=1}^{\infty} E_n\right) \leq \sum_{n=1}^{\infty} \mu^*(E_n)$ for every $\{E_n\}_{n \in \N} \subset \Pset(X)$.
\end{enumerate}
\end{definition}

\begin{remark}
If $\mu$ is a measure on $\Pset(X)$, then $\mu$ is also an outer measure.
\end{remark}

Now let $K \subseteq \Pset(X)$ with $\emptyset \in K$. We want to say that $K$ is the collection of elementary sets having a certain "measure", given by $\nu$ (called elementary measure).
Let $\nu: K \to \Rpext$ be a function such that $\nu(\emptyset) = 0$.

Define the function $\mu^*: \Pset(X) \to \Rpext$ by
\[
\mu^*(E) \coloneqq \inf\left\{\sum_{n=1}^{\infty} \nu(I_n) : E \subseteq \bigcup_{n=1}^{\infty} I_n, \{I_n\} \subset K\right\}
\]
if $E \subseteq X$ can be covered by a countable union of sets $I_n \in K$, and $\mu^*(E) \coloneqq \infty$ otherwise.

\begin{theorem}
$\mu^*$ is an outer measure on $X$.
\end{theorem}

\begin{proof}
\begin{enumerate}
    \item Since $\emptyset \in K$, we have $\mu^*(\emptyset) \leq \nu(\emptyset) = 0 \Rightarrow \mu^*(\emptyset) = 0$.

    \item \textbf{Monotonicity}: If $E_1 \subseteq E_2$, then any countable cover of $E_2$ is also a countable cover of $E_1$. From the definition of $\mu^*$, it follows that $\mu^*(E_1) \leq \mu^*(E_2)$.
    If $E_2$ does not have a countable cover, then $\mu^*(E_1) \leq \mu^*(E_2) = \infty$.

    \item \textbf{$\sigma$-subadditivity}: We need to show $\mu^*\left(\bigcup_{n=1}^{\infty} E_n\right) \leq \sum_{n=1}^{\infty} \mu^*(E_n)$.
    This is obvious if $\sum_{n=1}^{\infty} \mu^*(E_n) = \infty$.
    Suppose $\sum_{n=1}^{\infty} \mu^*(E_n) < \infty$. Then every $\mu^*(E_n) < \infty$.
    By the definition of $\mu^*$, for every $\varepsilon > 0$ and $n \in \N$, there exists $\{I_{n,k}\} \subseteq K$ such that
    $E_n \subseteq \bigcup_{k=1}^{\infty} I_{n,k}$ and $\mu^*(E_n) + \varepsilon/2^n > \sum_{k=1}^{\infty} \nu(I_{n,k})$.

    Since $\bigcup_{n=1}^{\infty} E_n \subseteq \bigcup_{n,k=1}^{\infty} I_{n,k}$ and $\{I_{n,k}\} \subseteq K$, it follows that
    \[
    \mu^*\left(\bigcup_{n=1}^{\infty} E_n\right) \leq \sum_{n,k=1}^{\infty} \nu(I_{n,k}) < \sum_{n=1}^{\infty} [\mu^*(E_n) + \varepsilon/2^n] = \sum_{n=1}^{\infty} \mu^*(E_n) + \varepsilon.
    \]
    Since $\varepsilon$ is arbitrary, the conclusion follows.
\end{enumerate}
\end{proof}

\begin{remark}
An outer measure is a function defined on all subsets, satisfying subadditivity. It is used to construct measures via the Carathéodory extension theorem.
\end{remark}

\clearpage
\subsection{Generation of a Measure}
Let $\mu^*$ be an outer measure on $X$.

\begin{definition}
A set $E \subseteq X$ is said to be \textbf{$\mu^*$-measurable} if for every $Z \subseteq X$, we have
\[
\mu^*(Z) = \mu^*(Z \cap E) + \mu^*(Z \cap E^C).
\]
This is called the \textbf{Carathéodory condition}.
\end{definition}

\begin{remark}
The Carathéodory condition is the crucial definition for moving from an outer measure to a true measure. An outer measure is defined for all subsets of $X$ but is only countably additive on certain "nice" sets. A set $E$ is deemed "nice" or measurable if it cleanly splits every other set $Z$ into two parts whose outer measures add up correctly. This condition ensures that the measure will be additive when we restrict it to the family of measurable sets.

If we take $X = Z$, we have $\mu^*(X) = \mu^*(E) + \mu^*(E^C)$, so $\mu^*(E) = \mu^*(X) - \mu^*(E^C)$. This can be seen as an inner measure.
\end{remark}

\begin{lemma}
$E \subseteq X$ is $\mu^*$-measurable if and only if $\mu^*(Z) \geq \mu^*(Z \cap E) + \mu^*(Z \cap E^C)$ for every $Z \subseteq X$.
\end{lemma}

\begin{proof}
It is enough to show that for every $E \subseteq X$, $Z \subseteq X$ we have $\mu^*(Z) \leq \mu^*(Z \cap E) + \mu^*(Z \cap E^C)$, because then equality follows.
Since $Z = Z \cap X = Z \cap (E \cup E^C) = (Z \cap E) \cup (Z \cap E^C)$, by subadditivity of $\mu^*$ we get
\[
\mu^*(Z) \leq \mu^*(Z \cap E) + \mu^*(Z \cap E^C).
\]
\end{proof}

\begin{lemma}
If $\mu^*(E) = 0$, then $E$ is $\mu^*$-measurable.
\end{lemma}

\begin{proof}
For every $Z \subseteq X$, by monotonicity of $\mu^*$:
\[
\mu^*(Z \cap E) + \mu^*(Z \cap E^C) \leq \mu^*(E) + \mu^*(Z) = 0 + \mu^*(Z).
\]
So $\mu^*(Z) \geq \mu^*(Z \cap E) + \mu^*(Z \cap E^C)$. By the preceding lemma, the Carathéodory condition is fulfilled; hence, $E$ is $\mu^*$-measurable.
\end{proof}


\subsection{Lebesgue Measure}
Let $\LL \coloneqq \{E \subseteq X : E \text{ is } \mu^*\text{-measurable}\}$.

\begin{theorem}
Let $\mu^*$ be an outer measure on $X$. Then
\begin{enumerate}
    \item $\LL$ is a $\sigma$-algebra,
    \item The restriction of $\mu^*$ to $\LL$, denoted $\mu = \mu^*|_{\LL}$, is a complete measure on $\LL$.
\end{enumerate}
\end{theorem}

\subsection{Lebesgue Measure in $\R$}
Take $X = \R$. We choose $(K, \nu)$ with $K$ the intervals and $\nu$ their length.

Let $I$ be the family of open bounded intervals:
\[
I \coloneqq \{(a,b) : a,b \in \R, a \leq b\}.
\]
So $\emptyset \in I$ when $a = b$.

The elementary measure $\lambda_0: I \to \Rp$:
\[
\lambda_0(\emptyset) \coloneqq 0, \quad \lambda_0((a,b)) \coloneqq b - a.
\]

From $(I, \lambda_0)$, we generate $\lambda^*$ outer measure on $\R$:
\[
\lambda^*(E) \coloneqq \inf\left\{\sum_{n=1}^{\infty} \lambda_0(I_n) : E \subseteq \bigcup_{n=1}^{\infty} I_n, \{I_n\} \subset I\right\}
\]
for every $E \subseteq \R$ that can be covered by a countable union of open bounded intervals, and $\lambda^*(E) \coloneqq \infty$ otherwise.

\begin{definition}
$\lambda^*$ generated by $(I, \lambda_0)$ is called the \textbf{outer Lebesgue measure} on $\R$.
$\lambda^*$-measurable sets are called \textbf{Lebesgue measurable sets}.
The corresponding $\sigma$-algebra $\LL(\R)$ is called the \textbf{Lebesgue $\sigma$-algebra}.
The measure $\lambda \coloneqq \lambda^*|_{\LL(\R)}$ is called the \textbf{Lebesgue measure} on $\R$, and it is complete.
\end{definition}

We want to study concretely what is the Lebesgue measure and $\sigma$-algebra.

\begin{remark}
Consider the interval $(a,b)$. The outer measure is $b-a$. We can consider the covering of $(a,b)$ made just by $I = (a,b)$:
$\lambda^*((a,b)) \leq \lambda_0((a,b)) = b - a$.

Then we see $(a,b) \subseteq \bigcup_{n=1}^{\infty} I_n = \Omega$ an open set. Since it is open, $\Omega \supseteq [a,b] \Rightarrow \Omega \supseteq (a+\varepsilon, b-\varepsilon)$. The elementary measure of this covering:
$b - a - 2\varepsilon = \lambda_0((a+\varepsilon, b-\varepsilon)) \leq \sum_{n=1}^{\infty} \lambda_0(I_n)$.

Passing to the infimum: $b - a \leq \lambda^*((a,b))$. So we have that $\lambda^*((a,b)) = b - a$.
\end{remark}

\begin{theorem}
Any countable subset $E \subseteq \R$ is Lebesgue measurable and $\lambda(E) = 0$.
\end{theorem}

\begin{proof}
First consider $\{a\}$ for $a \in \R$. $\{a\}$ can be covered by $\{a\} \subseteq (a-\varepsilon, a+\varepsilon)$. Then $\lambda^*(\{a\}) \leq \lambda_0((a-\varepsilon, a+\varepsilon)) = 2\varepsilon$. Since $\varepsilon$ is arbitrary, $\lambda^*(\{a\}) = 0$. So $\{a\}$ is measurable and $\lambda(\{a\}) = 0$.

Now consider $E = \bigcup_{n=1}^{\infty} \{a_n\}$ with $a_n \in \R$. Each set $\{a_n\}$ is measurable. This is a countable union of sets, so $E \in \LL$. Also, $\lambda^*$ is sub-additive; hence,
\[
\lambda^*(E) = \lambda^*\left(\bigcup_{n=1}^{\infty} \{a_n\}\right) \leq \sum_{n=1}^{\infty} \lambda^*(\{a_n\}) = 0.
\]
Then $E \in \LL$ and $\lambda^*(E) = \lambda(E) = 0$.
\end{proof}

\begin{remark}
The converse is not true.
\end{remark}

\clearpage
\begin{theorem}
$\B(\R) \subseteq \LL(\R)$. Every Borel set is Lebesgue measurable.
\end{theorem}

\begin{proof}
\begin{enumerate} [label=\arabic*.]
    \item Show that for every $a \in \R$, $(a, \infty) \in \LL(\R)$.

    Let $A \subseteq \R$ be any set. We assume that $a \notin A$; otherwise, we replace $A$ by $A \setminus \{a\}$, which leaves the outer measure unchanged.
    We must show that $\lambda^*(A) \geq \lambda^*(A \cap (-\infty, a)) + \lambda^*(A \cap (a, +\infty))$.

    Since $\lambda^*(A)$ is defined as an infimum, to check this inequality it is necessary and sufficient to show that for any sequence $\{I_k\} \subset I$ that covers $A$, we have
    \[
    \sum_{k=1}^{\infty} \lambda_0(I_k) \geq \lambda^*(A_1) + \lambda^*(A_2)
    \]
    where $A_1 = A \cap (-\infty, a)$ and $A_2 = A \cap (a, +\infty)$.

    For each $k \in \N$, define $I_k' = I_k \cap (-\infty, a)$ and $I_k'' = I_k \cap (a, +\infty)$. Then $I_k'$ and $I_k''$ are disjoint intervals and $\lambda_0(I_k) = \lambda_0(I_k') + \lambda_0(I_k'')$.

    $\{I_k'\}$ is a countable cover of $A_1$ and $\{I_k''\}$ is a countable cover of $A_2$. Therefore,
    \[
    \lambda^*(A_1) \leq \sum_{k=1}^{\infty} \lambda_0(I_k'), \quad \lambda^*(A_2) \leq \sum_{k=1}^{\infty} \lambda_0(I_k'')
    \]
    Hence,
    \[
    \lambda^*(A_1) + \lambda^*(A_2) \leq \sum_{k=1}^{\infty} \lambda_0(I_k') + \sum_{k=1}^{\infty} \lambda_0(I_k'') = \sum_{k=1}^{\infty} [\lambda_0(I_k') + \lambda_0(I_k'')] = \sum_{k=1}^{\infty} \lambda_0(I_k).
    \]

    \item Show that for any open set $\Omega \subseteq \R$, $\Omega \in \LL(\R)$.

    From Step 1, $(a, \infty) \in \LL(\R) \Rightarrow (a, \infty)^C = (-\infty, a] \in \LL(\R)$.
    But also $(a, b) = (a, \infty) \cap (-\infty, b)$. These two belong to $\LL$, so also $(a, b)$ belongs to $\LL$.

    Also, we know that $\Omega = \bigcup_{\substack{r,s \in \mathbb{Q} \\ r < s \\ (r,s) \in \Omega}} (r,s)$, so $\Omega \in \LL(\R)$.

    \item Show that $\B(\R) \subseteq \LL(\R)$.

    $\B(\R)$ is the smallest $\sigma$-algebra which contains the collection $\G$ of all open sets of $\R$.
    Therefore, if another $\sigma$-algebra $\A$ contains $\G$, it follows that $\B(\R) \subseteq \A$.
    Since $\LL(\R)$ contains all open sets, we have $\LL(\R) \supseteq \B(\R)$.
\end{enumerate}
\end{proof}

\begin{remark}
From before we obtained $(a,b) \in \LL(\R)$ and $\lambda((a,b)) = b - a$.
\end{remark}

\begin{remark}
$\B(\R) \subsetneq \LL(\R)$, so it is strictly contained.
\end{remark}

\clearpage
\begin{theorem}
$(\R, \LL(\R), \lambda)$ is the completion of $(\R, \B(\R), \lambda|_{\B(\R)})$.
\end{theorem}

\begin{remark}
Completion is a process of making a measure space complete. This theorem states that the Lebesgue $\sigma$-algebra is exactly what you get if you start with all Borel sets and then throw in every subset of every Borel set that has measure zero. This is often the most practical way to think about what a Lebesgue measurable set is.
\end{remark}

\subsection{What is a Lebesgue Measurable Set?}
All Borel measurable sets are Lebesgue measurable, but we want to understand better how a Lebesgue measurable set is related to open and closed sets.

\begin{lemma}[Excision property]
If $A \in \LL(\R)$ with $\lambda^*(A) < \infty$ and $A \subseteq B$, then
\[
\lambda^*(B \setminus A) = \lambda^*(B) - \lambda^*(A).
\]
\end{lemma}

\begin{proof}
Since $A \in \LL(\R)$, the Carathéodory condition is satisfied, so (with $Z = B$):
\[
\lambda^*(B) = \lambda^*(B \cap A) + \lambda^*(B \cap A^C) = \lambda^*(A) + \lambda^*(B \setminus A).
\]
And the conclusion follows.
\end{proof}

\clearpage
\begin{theorem}[Regularity of Lebesgue measure]
Let $E \subseteq \R$. The following statements are equivalent:
\begin{enumerate}
    \item $E \in \LL(\R)$,
    \item For every $\varepsilon > 0$, there exists $A \subseteq \R$ open such that $E \subseteq A$ and $\lambda^*(A \setminus E) < \varepsilon$,
    \item There exists $G \subseteq \R$ of class $G_\delta$ (countable intersection of open sets) such that $E \subseteq G$ and $\lambda^*(G \setminus E) = 0$,
    \item For every $\varepsilon > 0$, there exists $C \subseteq \R$ closed such that $C \subseteq E$ and $\lambda^*(E \setminus C) < \varepsilon$,
    \item There exists $F \subseteq \R$ of class $F_\sigma$ (countable union of closed sets) such that $F \subseteq E$ and $\lambda^*(E \setminus F) = 0$.
\end{enumerate}
\end{theorem}

\begin{proof}
Let's consider only the open sets: $1 \Rightarrow 2 \Rightarrow 3 \Rightarrow 1$.

\begin{itemize}[leftmargin=5em]
    \item[$(1) \Rightarrow (2)$] $E \in \LL(\R)$ and we assume that $\lambda(E) < \infty$.
    By the definition of outer measure, for every $\varepsilon > 0$, there exists $\{I_k\} \subset I$ of open bounded intervals which covers $E$ and for which
    \[
    \sum_{k=1}^{\infty} \lambda_0(I_k) < \lambda^*(E) + \varepsilon.
    \]
    Define the set $O \coloneqq \bigcup_{k=1}^{\infty} I_k$. Then $O$ is open and $E \subseteq O$. Therefore,
    \[
    \lambda^*(O) \leq \sum_{k=1}^{\infty} \lambda_0(I_k) < \lambda^*(E) + \varepsilon.
    \]
    This implies that $\lambda^*(O) - \lambda^*(E) < \varepsilon$.
    Given $E \in \LL(\R)$ and $\lambda^*(E) < \infty$, we can say $\lambda^*(O \setminus E) = \lambda^*(O) - \lambda^*(E) < \varepsilon$.

    \item[$(2) \Rightarrow (3)$] For every $k \in \N$, choose $O_k \supseteq E$ open set such that $\lambda^*(O_k \setminus E) < 1/k$.
    Define $G \coloneqq \bigcap_{k=1}^{\infty} O_k$. Then $G \in G_\delta$ and $G \supseteq E$.
    Moreover, for every $k \in \N$, $G \setminus E \subseteq O_k \setminus E$. By monotonicity,
    \[
    \lambda^*(G \setminus E) \leq \lambda^*(O_k \setminus E) < 1/k.
    \]
    For $k \to \infty$, we have $\lambda^*(G \setminus E) = 0$.

    \item[$(3) \Rightarrow (1)$] $G \setminus E \in \LL(\R)$ since $\lambda^*(G \setminus E) = 0$.
    Now $G \in \LL(\R)$ since $G \in G_\delta \subset \B(\R) \subset \LL(\R)$.
    Therefore, $E = G \cap (G \setminus E)^C \in \LL(\R)$.
\end{itemize}
\end{proof}

\begin{remark}
This Regularity Theorem is incredibly powerful. It tells us that a set is measurable if and only if it can be approximated arbitrarily closely from the outside by open sets (which are too big by only a measure $\varepsilon$) and from the inside by closed sets (which are too small by only a measure $\varepsilon$). Furthermore, it states that every measurable set $E$ is essentially a "nice" Borel set (a $G_\delta$ or $F_\sigma$) with a negligible set attached or removed. This bridges the abstract definition of measurability with a more intuitive geometric understanding.
\end{remark}

\clearpage
\subsection{Non-Measurable Sets}

\begin{theorem}[Vitali]
Any Lebesgue measurable set $E \in \LL(\R)$ with $\lambda(E) > 0$ contains a subset that is not Lebesgue measurable.
\end{theorem}

\begin{theorem}
There exist disjoint sets $A, B \subset \R$ for which $\lambda^*(A \cup B) < \lambda^*(A) + \lambda^*(B)$.
\end{theorem}

\begin{proof}
Assume by contradiction that $\lambda^*(A \cup B) = \lambda^*(A) + \lambda^*(B)$ for any $A, B \subset \R$ that are disjoint ($A \cap B = \emptyset$).

For every $E, Z \subseteq \R$:
\[
\lambda^*(Z \cap E) + \lambda^*(Z \cap E^c) = \lambda^*((Z \cap E) \cup (Z \cap E^c)) = \lambda^*(Z).
\]
If this is true for every $E \subseteq \R$, it fulfills the Carathéodory condition, so any $E \in \LL(\R)$. This is not possible because there exists the Vitali set.
\end{proof}

\begin{remark}
This theorem highlights the need for the Carathéodory condition. The outer measure, by itself, is not a measure because it fails countable additivity on the entire power set of $\R$. We must restrict it to the smaller $\sigma$-algebra $\LL(\R)$ where additivity holds.
\end{remark}

\subsection{Lebesgue Measure in $\R^N$}
Take $X = \R^N$. Take $K$ as the family of $N$-dimensional open intervals (rectangles):
\[
I^N \coloneqq \left\{\prod_{k=1}^{N} (a_k, b_k) : a_k, b_k \in \R \text{ and } a_k \leq b_k\right\}.
\]
Take $\nu$ as the elementary volume function $\lambda_0^N: I^N \to [0, \infty)$:
\[
\lambda_0^N(\emptyset) \coloneqq 0, \quad \lambda_0^N\left(\prod_{k=1}^{N} (a_k, b_k)\right) \coloneqq \prod_{k=1}^{N} (b_k - a_k).
\]

$(K, \nu) = (I^N, \lambda_0^N)$

\begin{itemize}
    \item This generates the $N$-dimensional Lebesgue outer measure $\lambda^{* ,N}$,
    \item The $\lambda^{* ,N}$-measurable sets form, with the Carathéodory condition, the $N$-dimensional Lebesgue $\sigma$-algebra $\LL(\R^N)$,
    \item The restriction $\lambda^N \coloneqq \lambda^{* ,N}|_{\LL(\R^N)}$ is the $N$-dimensional Lebesgue measure,
    \item The space $(\R^N, \LL(\R^N), \lambda^N)$ is a complete measure space.
\end{itemize}