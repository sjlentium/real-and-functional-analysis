\section{Types of Convergence}

\subsection{Convergences}
Let $\{f_n\} \subseteq \M(X,\A)$ with $f_n : X \to \R$ and $f : X \to \Rext$.

\subsubsection{Pointwise Convergence}
$f_n \to f$ pointwise in $X$ as $n \to \infty$ if and only if
\[
\forall x \in X, \quad f_n(x) \to f(x) \quad \text{as } n \to \infty.
\]

\subsubsection{Uniform Convergence}
$f_n \to f$ uniformly in $X$ as $n \to \infty$ if and only if
\[
\sup_{x \in X} |f_n(x) - f(x)| \to 0 \quad \text{as } n \to \infty.
\]

\subsubsection{Almost Everywhere Convergence}
$f_n \to f$ a.e. in $X$ as $n \to \infty$ if and only if
\[
\{x \in X : f_n(x) \not\to f(x)\}^c \in \NN_\mu.
\]

\subsubsection{Convergence in $L^1$}
For $\{f_n\} \subseteq L^1$ and $f \in L^1$, we say $f_n \to f$ in $L^1$ as $n \to \infty$ if and only if
\[
d(f_n, f) \to 0 \quad \text{as } n \to \infty,
\]
which is equivalent to
\[
\int_X |f_n - f| \, d\mu \to 0 \quad \text{as } n \to \infty.
\]

\subsubsection{Convergence in $L^\infty$}
For $\{f_n\} \subseteq L^\infty$ and $f \in L^\infty$, we say $f_n \to f$ in $L^\infty$ as $n \to \infty$ if and only if
\[
d(f_n, f) \to 0 \quad \text{as } n \to \infty,
\]
which is equivalent to
\[
\esssup_X |f_n - f| \to 0 \quad \text{as } n \to \infty.
\]

\subsubsection{Convergence in Measure}
We say that $f_n \to f$ in measure as $n \to \infty$ if
\[
\forall \varepsilon > 0, \quad \mu(\{x \in X : |f_n(x) - f(x)| \geq \varepsilon\}) \to 0 \quad \text{as } n \to \infty.
\]

\clearpage
\subsection{Theorems}

\begin{theorem}
Let $\mu(X) < \infty$ and let $f_n, f \in \M(X,\A)$ be finite a.e. in $X$. If $f_n \to f$ a.e. in $X$, then $f_n \to f$ in measure.
\end{theorem}

\begin{remark}
When $\mu(X) = \infty$, convergence a.e. $\not\Rightarrow$ convergence in measure.
\end{remark}

\begin{example}
Let $f_n : \R \to \R$ with $\lambda(\R) = \infty$ and define $f_n := \chi_{[n, +\infty)}$. Then $f_n \to 0$ pointwise in $\R$, but
\[
\mu(\{f_n \geq 1/2\}) = \mu([n, +\infty)) = +\infty,
\]
so $f_n \not\to 0$ in measure.
\end{example}

\begin{remark}
Convergence in measure $\not\Rightarrow$ convergence a.e. This can be shown using the counterexample of the Typewriter sequence.
\end{remark}

\begin{theorem}
Let $f_n, f \in \M(X,\A)$ be finite a.e. in $X$. If $f_n \to f$ in measure, then there exists a subsequence $\{f_{n_k}\}$ such that $f_{n_k} \to f$ a.e. in $X$ as $k \to \infty$.
\end{theorem}

\begin{theorem}
Let $f_n, f \in L^1(X,\A,\mu)$. If $f_n \to f$ in $L^1$, then $f_n \to f$ in measure.
\end{theorem}

\begin{proof}
Suppose, by contradiction, that $f_n \not\to f$ is in measure. Then there exist $\varepsilon > 0$ and $\sigma > 0$ such that
\[
\mu(\{|f_n - f| \geq \varepsilon\}) \geq \sigma \quad \text{for infinitely many } n \in \N.
\]
Thus,
\[
\int_X |f_n - f| \, d\mu \geq \int_{\{|f_n - f| \geq \varepsilon\}} |f_n - f| \, d\mu \geq \int_{\{|f_n - f| \geq \varepsilon\}} \varepsilon \, d\mu = \varepsilon \mu(\{|f_n - f| \geq \varepsilon\}) \geq \varepsilon \sigma
\]
for infinitely many $n \in \N$, which implies $f_n \not\to f$ in $L^1$. This is a contradiction.
\end{proof}

\begin{remark}
If $f_n \to f$ in measure, this $\not\Rightarrow f_n \to f$ in $L^1$.
\end{remark}

\begin{example}
Let $f_n(x) = n \chi_{[0, 1/n]}(x)$ for $x \in [0,1]$ and $f = 0$. Then $f_n \to f$ a.e. in $[0,1]$. Since $\lambda([0,1]) = 1 < \infty$, we have $f_n \to 0$ in measure. However,
\[
\int_{[0,1]} |f_n - 0| \, d\lambda = \int_0^1 f_n \, dx = \int_0^{1/n} n \, dx = n \cdot \frac{1}{n} = 1,
\]
so $f_n \not\to f$ in $L^1([0,1])$.
\end{example}

\clearpage
\begin{corollary}
If $f_n \to f$ in $L^1$, then there exists a subsequence $\{f_{n_k}\}$ such that $f_{n_k} \to f$ a.e. in $X$ as $k \to \infty$.
\end{corollary}

\begin{remark}
To check if $\{f_n\}$ converges in $L^1$, first study the limit a.e. If $f := \lim_{n \to \infty} f_n$ exists a.e. in $X$, then $f$ is the candidate limit for the $L^1$ convergence.
\end{remark}

\subsection{Typewriter Sequence (Rademacher Sequence)}

Determine $k \in \N$ such that $2^k \leq n \leq 2^{k+1}$ (so $k = \lfloor \log_2 n \rfloor$). Define
\[
f_n(x) := \chi_{\left[\frac{n - 2^k}{2^k}, \frac{n - 2^k + 1}{2^k}\right]}(x), \quad I_n := \left[\frac{n - 2^k}{2^k}, \frac{n - 2^k + 1}{2^k}\right].
\]

For example:
\begin{itemize}
\item $n = 1$, $k = 0$: $I_1 = [0,1]$
\item $n = 2$, $k = 1$: $I_2 = [0, 1/2]$
\item $n = 3$, $k = 1$: $I_3 = [1/2, 1]$
\item $n = 4$, $k = 2$: $I_4 = [0, 1/4]$
\item $n = 5$, $k = 2$: $I_5 = [1/4, 1/2]$
\end{itemize}

We observe that for every $x \in [0,1]$, there exists a countable set $J \subset \N$ such that $x \in I_n$ for all $n \in J$, and $x \notin I_n$ for all $n \in \N \setminus J$. It follows that
\[
f_n(x) = 1 \text{ for infinitely many } n \in \N, \quad f_n(x) = 0 \text{ for infinitely many } n \in \N.
\]
Thus, the limit $\lim_{n \to \infty} f_n(x)$ does not exist. Pointwise convergence does not occur, and this holds for every $x$, so we cannot have a.e. pointwise convergence.

However,
\[
\int_{[0,1]} f_n \, d\lambda = \int_0^1 f_n \, dx = \int_{\frac{n - 2^k}{2^k}}^{\frac{n - 2^k + 1}{2^k}} 1 \, dx = \frac{1}{2^k} \to 0 \quad \text{as } n \to \infty \ (k \to \infty).
\]
Hence, $f_n \to 0$ in $L^1([0,1])$ and $f_n \to 0$ in measure in $[0,1]$.